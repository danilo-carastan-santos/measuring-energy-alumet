\documentclass[a4paper,10pt]{article}

% Package defining the style for exercice sheets (style inspired from lsr-exo)
%
% First version: Septembre 2015
% Current version: Septembre 2015
%
%\NeedsTeXFormat{LaTeX2e}
%\ProvidesPackage{tpstyle}[2015/10/05 Package for the exercices of Thomas]
%
\usepackage[utf8]{inputenc}
\usepackage{fancyhdr}
\usepackage{ifthen}
\usepackage{graphicx}
\usepackage{fullpage}
\usepackage{calc}
\usepackage{amsmath}
%listings: traitement de code source...
\usepackage{listings}
\usepackage[textwidth=17mm]{todonotes}


% to highlight the comments and questions
\newcounter{cpt-note}
\newcounter{cpt-question}
\newcommand{\Note}[2][inline]{\vspace{0.2cm}\todo[bordercolor=brown!70,backgroundcolor=white,#1]{\addtocounter{cpt-note}{1}
    \textbf{Note \arabic{cpt-note}} #2}\vspace{0.2cm}}

\newcommand{\Question}[2][inline]{\vspace{0.1cm}\todo[bordercolor=blue,backgroundcolor=white,#1]{\addtocounter{cpt-question}{1}
    \sf \textbf{Question \arabic{cpt-question}} #2}\vspace{0.2cm}}

\newcommand{\Important}[2][inline]{\vspace{0.1cm}\todo[backgroundcolor=red!20,#1]{\sf \textbf{Important} #2}\vspace{0.1cm}}

\newcommand{\Reponse}[2][inline]{\todo[bordercolor=gray, backgroundcolor=gray!10,#1]{\sf \textbf{Réponse} #2}\vspace{0.1cm}}


% taille de la barre de l'en-tete
\newcommand{\tdlinewidth}{.1pt}


\newcommand{\tdsection}{?}
\newcommand{\tdyear}{?}
\newcommand{\tdaffiliation}{?}
\newcommand{\tdtitle}{?}
\newcommand{\tdnumber}{?}
\newcommand{\tdlongtitle}{?}


%commandes pour changer les options ci-dessus
\newcommand{\deftdsection}[1]{\renewcommand\tdsection{#1}}
\newcommand{\deftdyear}[1]{\renewcommand\tdyear{#1}}
\newcommand{\deftdaffiliation}[1]{\renewcommand\tdaffiliation{#1}}
\newcommand{\deftdtitle}[1]{\renewcommand\tdtitle{#1}}
\newcommand{\deftdnumber}[1]{\renewcommand\tdnumber{#1}}
\newcommand{\deftdlongtitle}[1]{\renewcommand\tdlongtitle{#1}}

%afficher 'Exercice' lorsqu'on fait une nouvelle section
%\renewcommand{\thesection}{Exercice \@arabic\c@section{} :}


%affichage normal pour tous les niveaux hierarchiques inferieurs
%% \renewcommand{\thesubsection}{Question
%%   \@arabic\c@section.\@arabic\c@subsection{}}

%\renewcommand*{\familydefault}{\sfdefault}


%mise en page: ajoute un peu d'espace sous l'en-tete
\setlength{\headheight}{.5cm}
\setlength{\headsep}{.5cm}


\renewcommand{\maketitle}{%
  ~\\
  \begin{center}
    {\huge \tdtitle{} \tdnumber{} 
    }%
  \end{center}
  \vspace{.3cm}
  \begin{center}
    {\LARGE \textbf \tdlongtitle 
    }%
  \end{center}
  \vspace{.5cm}
}



\AtBeginDocument{%
  \pagestyle{fancy}%
  \fancyhf{}%
  %mise en place de l'en-tete
  \lhead{\small \tdsection{} -- \tdyear}%
  \rhead{\small \tdaffiliation\hspace{.5cm}\thepage}%
  \renewcommand{\headrulewidth}{\tdlinewidth}%
}

%\usepackage{url}

\usepackage{hyperref}

\usepackage{soul}

\usepackage[utf8]{inputenc}
%\usepackage[T1]{fontenc}
\usepackage[french]{babel}

% Warning/info blocks
\usepackage{pifont,mdframed}
\usepackage{emoji}

\newenvironment{warning}
  {\par\begin{mdframed}[linewidth=2pt,linecolor=orange]%
    \begin{list}{}{\leftmargin=1cm
                   \labelwidth=\leftmargin}\item[\emoji{warning}]}
  {\end{list}\end{mdframed}\par}

  \newenvironment{note}
  {\par\begin{mdframed}[linewidth=2pt,linecolor=blue]%
    \begin{list}{}{\leftmargin=1cm
                   \labelwidth=\leftmargin}\item[\ding{45}]}
  {\end{list}\end{mdframed}\par}

  \newenvironment{information}
  {\par\begin{mdframed}[linewidth=2pt,linecolor=blue]%
    \begin{list}{}{\leftmargin=1cm
                   \labelwidth=\leftmargin}\item[\emoji{information}]}
  {\end{list}\end{mdframed}\par}

%% Jupyter style code block
\usepackage{listings}
\usepackage{tcolorbox}
\usepackage{xcolor}

% Define Jupyter notebook-style colors
\definecolor{notebookbg}{rgb}{0.95,0.95,0.92} % Light gray background
\definecolor{codebg}{rgb}{0.98,0.98,0.98}     % Lighter background for the code area
\definecolor{codeborder}{rgb}{0.6,0.6,0.6}    % Light gray border
\definecolor{codefont}{rgb}{0.1,0.1,0.1}      % Dark font color

% Define the Jupyter code style using tcolorbox and listings
\newtcolorbox{jupytercode}[1][]{colback=codebg, colframe=codeborder, left=5pt, right=5pt, top=5pt, bottom=5pt, boxrule=0.4pt, sharp corners, enhanced, #1}

\lstset{
    basicstyle=\ttfamily\footnotesize, % Monospaced font
    keywordstyle=\bfseries\color{blue}, % Keywords in blue and bold
    commentstyle=\itshape\color{green!60!black}, % Comments in italic green
    stringstyle=\color{red}, % Strings in red
    showstringspaces=false,
    backgroundcolor=\color{codebg}, % Background color
    frame=none, % No border frame by default
    tabsize=4,
    breaklines=true,
    numbers=left, % Line numbers on the left
    numberstyle=\tiny\color{gray}, % Style of the line numbers
}

  


\deftdsection{M2 MIAGE -- Éco-Conception Web}
\deftdyear{\the\year}
\deftdaffiliation{UFR IM2AG}
\deftdtitle{TP}
\deftdnumber{}
\deftdlongtitle{Observation et outils de mesure énergétique}

\newtheorem{question}{Question}


\usepackage{etoolbox}
\makeatletter
\preto{\@verbatim}{\topsep=0pt \partopsep=0pt }
\makeatother


\begin{document}


\renewcommand{\labelitemi}{$\bullet$}

\maketitle


%% L'objectif de ce TP est de se familiariser avec l'utilisation de
%% Git.

\section*{Préambule}
Tous les outils de mesure d'énergie nécessitent soit des droits de
superutilisateur sur la machine, soit du matériel supplémentaire (wattmètre).
Les \textit{hyperscalers} (AWS, Google Cloud, etc.) ne vous donnent pas accès à
de tels outils. Néanmoins, le contenu présenté dans ce TP devrait être utile
lors de (i) la phase de développement, où l'application s'exécute sur un
ordinateur local, ou (ii) dans un cloud privé, où un contrôle plus approfondi du
système/matériel est possible.


\section{Objectifs du TP}

\begin{itemize}
  \item Utiliser des outils de monitoring d’énergie
  \item Visualiser quels composants sont inclus dans un outil de mesure
  \item Étudier l’impact du nombre de ressources dans la consommation énergétique
  \item Étudier l’impact du choix du datacentre la consommation énergétique 
\end{itemize}



%% \section{Prise en main}

%% %% De nombreux tutoriels sur Git sont disponibles sur Internet. Pour un
%% %% premier contact avec Git, nous nous proposons de faire le tutoriel
%% %% suivant (en anglais):

%% %% \begin{center}
%% %%   \url{https://learngitbranching.js.org/}
%% %% \end{center}

%% %% % http://try.github.io

%% %% Prenez le temps de bien comprendre ce qu'il se passe et de faire le
%% %% lien avec ce qui a été présenté en cours.


%% %% Si vous êtes allergiques à l'anglais, le tutoriel
%% %% \url{http://rogerdudler.github.io/git-guide/index.fr.html} peut être
%% %% une alternative.


%% Si vous voulez commencer le TP par un petit tutoriel en français, nous
%% vous suggérons le suivant: \url{http://rogerdudler.github.io/git-guide/index.fr.html}.


\section{Consommation électrique et impact environnemental}

\subsection{Qu'est-ce qui consomme ?}

L'ordinateur sur lequel tourne votre programme consomme de l'électricité pour
alimenter ses composants : processeur, barrettes de RAM, disque dur, carte
réseau, ventilateurs, périphériques USB, etc. Il y a également des pertes dans
l'alimentation électrique (rendement inférieur à 100\%) et dans les circuits
(perte sous forme de chaleur par effet Joule). La consommation des composants
dépend parfois de leur utilisation. C'est par exemple le cas pour le CPU, car
les CPUs modernes sont capables de s'éteindre partiellement lorsqu'ils sont peu
sollicités.

La consommation d'un logiciel donné n'est pas directement mesurable, mais on
peut l'estimer en attribuant la consommation du matériel au logiciel.
 
\begin{information}
    Pour avoir une attribution plausible, il faut bien sûr tenir compte de
    l'utilisation du matériel par le logiciel.

    Intuitivement, si on exécute deux programmes en même temps sur la même machine :
    \begin{itemize}
        \item un programme A qui ne fait que "dormir" (par exemple avec un \texttt{sleep(99999999)})
        \item un programme B de simulation numérique qui exécute en boucle des calculs matriciels
    \end{itemize}  
  
    On voit que le programme B devrait être "responsable" d'une plus grande part
    de la consommation électrique que le programme A.
  
    En connaissant la consommation du matériel et l'utilisation du matériel par
    le logiciel, on peut calculer une estimation de l'énergie consommée par
    chaque programme.
  
\end{information}
  

\subsection{Comment mesurer ?}

\begin{information}
    Pour cette partie, vous devez impérativement être sous Linux, de préférence
    Ubuntu 22.04 LTS (les postes de l'IM2AG correspondent parfaitement à cette
    contrainte).
\end{information}
  

Le plus simple est de se concentrer sur le CPU et la RAM, qui disposent
d'interfaces bas niveau permettant de lire leur consommation électrique à chaque
petit intervalle de temps $\Delta t$ (chez Intel, $\Delta t = 976 \mu s$). Il
existe plusieurs logiciels pour récupérer cette information, mais la plupart
souffrent d'erreurs d'implémentation ou de limitations conceptuelles
\footnote{Guillaume Raffin, Denis Trystram. Dissecting the software-based
measurement of CPU energy consumption: a comparative analysis. 2024.
\href{https://hal.science/hal-04420527}{⟨hal-04420527v2⟩}}.

Pour ce TP, nous allons utiliser un nouvel outil :
\href{https://alumet.dev/}{ALUMET} (Adaptive, Lightweight, Unified METrics).
Alumet propose un cadre modulaire permettant de mesurer tout type de matériel,
de récolter des informations sur le système, et de combiner les deux vers tout
type de sortie (fichier, base de données). Un système de plugins permet de
construire un outil de mesure ``sur mesure''. Alumet peut se déployer sur une
seule machine ou dans un système distribué (HPC classique ou Kubernetes par
exemple).

Vous allez utiliser Alumet pour mesurer la consommation d'énergie de votre
machine pendant que le scheduler s'exécute, et obtenir une estimation de la
consommation du scheduler lui-même.

Travail :
\begin{enumerate}
    \item  Aller lire le README du \href{https://github.com/alumet-dev/alumet}{dépôt
    GitHub}. N'hésitez pas à mettre une petite étoile \emoji{star} ! 
    \item Télécharger ALUMET à utiliser pour ce TP,
    c'est le fichier \texttt{alumet-local-agent}. 
    \item Vérifier qu'Alumet tourne :
\end{enumerate}

\begin{jupytercode}
    \begin{lstlisting}
        git clone 
        ./alumet-local-agent exec -- sleep 3
    \end{lstlisting}    
\end{jupytercode}

Vous devriez obtenir un fichier \texttt{alumet-output.csv} contenant des mesures.
Utilisez
\href{https://alumet-dev.github.io/user-book/installation/exec.html}{la
documentation d'Alumet} pour comprendre le contenu du fichier.

\subsection{Observer la consommation énergétique avec Alumet et \textt{stress}}

\begin{jupytercode}
    \begin{lstlisting}[language=Python]
        import subprocess
        from datetime import datetime
        import time
        import os
        
        NB_CPUS = 16
        PROC_TIME_SECONDS = 10
        SLEEP_TIME_SECONDS = 1
        
        ALUMET_RESULT_FILENAME = "../Results-CSV/"+datetime.now().strftime('%Y-%m-%d_%H:%M:%S')+"_alumet-output.csv"+".csv"
        
        alumet_command = ["./alumet-local-agent"]
        
        stress_command = ["./stress","-c",str(NB_CPUS),"-t",str(PROC_TIME_SECONDS)]
        
        ## Lancer Alumet et la commande stress
        
        alumet_start = time.time()
        ## Lancer perf stat et ne pas attendre sa fin
        alumet_pid = subprocess.Popen(alumet_command).pid
        
        ## Attendre un peu pour bien monitorer la commande stress
        time.sleep(SLEEP_TIME_SECONDS)    
        
        stress_start = time.time()
        ## Lancer la commande stress et attendre sa fin
        subprocess.run(stress_command)
        stress_end = time.time()
        
        ## Attendre un peu pour bien monitorer la commande stress
        time.sleep(SLEEP_TIME_SECONDS)
        
        ## Terminer Alumet en envoyant SIGINT (-2)
        ## Équivalent à Ctrl+c
        subprocess.run(["kill", "-2", str(alumet_pid)])
        
        ## Renommer le CSV de sortie d'Alumet
        os.rename("./alumet-output.csv", ALUMET_RESULT_FILENAME)
        print("Done")
    \end{lstlisting}
    \end{jupytercode}

\begin{information}
    Il se peut que vous obteniez des erreurs ou warnings à propos de priorité de
  threads. C'est dû à un manque de privilèges sur les machines de l'école.
  Ignorez-les, ça n'empêche pas Alumet de fonctionner.
  
  En cas d'erreur plus grave, appeler le professeur pour trouver une alternative.
\end{information}
  
\begin{enumerate}
    \item Utiliser le mode "exécution" d'Alumet (comme ci-dessus) pour obtenir
    des mesures lors de l'exécution de votre scheduler. Tracez une courbe (avec
    LibreOffice par exemple) de la consommation du CPU et de la RAM. Faites
    l'expérience pour différents algorithmes et fichiers d'entrée (pensez à
    copier les résultats à chaque fois ou à passer un paramètre pour changer le
    nom de la sortie, sinon le fichier CSV sera écrasé - voir `--help`).
    \item Comparer la consommation des différentes implémentations de
    scheduling.
    \item D'après vous, quel serait le meilleur compromis consommation d'énergie
    / solution optimale ou proche de l'optimale ?
\end{enumerate}


\subsection{Empreinte carbone}

Produire de l'électricité entraîne des emissions de gaz à effet de serre plus ou
moins importantes en fonction du mode de production. L'empreinte carbone d'un
datacentre dépend donc de ses sources d'électricité et, à moins qu'il ne soit
totalement indépendant du réseau, de sa position.

Travail :
\begin{itemize}
    \item Utiliser \href{https://app.electricitymaps.com/map}{ElectricityMaps}
    pour visualiser l'intensité carbone (en grammes d'équivalent CO2 par kWh
    d'électricité) de la consommation d'énergie par pays. Quelles sont les
    sources dites ``bas carbone'' ?
    \item Multiplier l'intensité carbone de la consommation d'électricité en
    France par la consommation du CPU lorsqu'il exécute votre scheduler pour
    obtenir une idée de son empreinte carbone en phase d'usage. Extrapoler à une
    année entière d'utilisation (24h/24 7j/7).
    \item (Question ouverte) Est-il toujours bénéfique d'avoir un scheduler qui
    consomme moins d'électricité ? % on pense bien sûr à l'effet rebond

\end{itemize}

\subsection{Un mot sur ce que nous n'avons pas mesuré}

La production de gaz à effet de serre n'est pas le seul impact de
l'informatique, et il n'y a pas que la phase d'usage qui compte. Il faut aussi
fabriquer, transporter, puis gérer la fin de vie des équipements. Même en phase
d'usage, la consommation électrique entraîne d'autres impacts : en plus de
l'empreinte carbone, on devrait évaluer la consommation d'eau, de métaux, etc.


\end{document}
